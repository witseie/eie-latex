\documentclass[11pt]{eie-cbo}

% assumes the use of pdfLaTeX

% comment out these two lines if you wish to use the standard (uglier) LaTeX font
	\RequirePackage[bitstream-charter]{mathdesign} % main font
	\RequirePackage{microtype}  % character protusion and font expansion

% specify the course code and name
\coursecode{ELEN????A}
\coursename{???????}

%	General Considerations:
% 1.	The CB&O is the design document and execution plan for the course.
% 2.	The CB&O is the property of the School and major change is subject to approval by the Curriculum Committee.
% 3.	The CB&O should not specify anything that cannot be implemented.

\begin{document}
\makeheads

%%%%%%%%%%%%%%%%%%%%%%%%%%%%%%%%%%%%%%%%%%%%%%%%%%%%%%%%%%%%%%%%%%%%%%%%%%%%%%%%
\section*{Academic Staff:}
\begin{tabular}{l}
<Name ???>															\\
Room: ???                          			\\
Tel: ???                      					\\
Email: \email{???@wits.ac.za}
\end{tabular}

%%%%%%%%%%%%%%%%%%%%%%%%%%%%%%%%%%%%%%%%%%%%%%%%%%%%%%%%%%%%%%%%%%%%%%%%%%%%%%%%
\section{Course Background}\label{sec:background}

% How the subject fits in the overall degree and field of engineering:  Max 100 words).

%%%%%%%%%%%%%%%%%%%%%%%%%%%%%%%%%%%%%%%%%%%%%%%%%%%%%%%%%%%%%%%%%%%%%%%%%%%%%%%%
\section{Course Objectives}\label{sec:objectives}
% What does the course contribute by way of readiness for further studies or work, development of knowledge areas and toward ELOs or final assessment of ELOs:  Max 100 words. 

%%%%%%%%%%%%%%%%%%%%%%%%%%%%%%%%%%%%%%%%%%%%%%%%%%%%%%%%%%%%%%%%%%%%%%%%%%%%%%%%
\section{Course Outcomes}\label{sec:outcomes}
On successful completion of this course, the student is able to:
\begin{enumerate}
	\item 
	\item 
	\item 
\end{enumerate}

% In a course that assesses outcomes at exit level, identify components containing one or more outcome that must be satisfied; this may be linked to sections of the Content Content (next section). Outcomes that are intended to satisfy or partially satisfy ECSA exit level outcomes must be flagged as such.
% One or more outcomes may be identified as a course component.


%%%%%%%%%%%%%%%%%%%%%%%%%%%%%%%%%%%%%%%%%%%%%%%%%%%%%%%%%%%%%%%%%%%%%%%%%%%%%%%%
\section{Course Content}\label{sec:content}
\begin{description}
		\item []
		\item []
		\item [] 		
\end{description}	

% List the knowledge elements, principles, concepts and skills or methods covered in the course. The entry in the Rules & Syllabuses: Faculty of Engineering and the Built Environment will be an abbreviated version of this section with a lead-in sentence from 2: course Objectives above. Do not cross reference to Rules & Syllabuses.

%%%%%%%%%%%%%%%%%%%%%%%%%%%%%%%%%%%%%%%%%%%%%%%%%%%%%%%%%%%%%%%%%%%%%%%%%%%%%%%%
\section{Prior Knowledge Assumed}\label{sec:prior}
The following prior knowledge is assumed on the part of students starting this course: 
% Insert principles, concepts, skills and methods, not course pre-requisites and co-requisites.

The prerequisites and co-requisites to register for this course are defined in the current \emph{Rules \& Syllabuses: Faculty of Engineering and the Built Environment}.

%%%%%%%%%%%%%%%%%%%%%%%%%%%%%%%%%%%%%%%%%%%%%%%%%%%%%%%%%%%%%%%%%%%%%%%%%%%%%%%%
\section{Assessment}\label{assessment}

\subsection{Formative Assessment}\label{formative}
% List or describe generally the formative assessment approach and elements in the course: Formative assessments do not contribute toward the final result but may be a SP requirement.

\pagebreak
\subsection{Summative Assessment}\label{summative}

%< List the summative assessment elements in the course and their contributions to the final result in the table below. Identify Assignments, projects, tests, design, build & test, …etc  and examinations.

\begin{table}[ht]\small\centering
\begin{tabular}{lccccl}\toprule
\multirow{2}{2cm}{Assessment Contributor}  & Duration & Component & \multirow{2}{1.5cm}{Method \& Weight} & Calculator & \multirow{2}{3.5cm}{Permitted Supporting Material}\\ 
 & (hours) & & & Type &  \\
\midrule
 &   &  &  &  &   \\
 &   &  &  &  &   \\
 &   &  &  &  &   \\
\bottomrule
\end{tabular}\label{tab:sum-contributors}
\end{table}

% Under Summative Assessment Contributor include a keyword such as examination, test, essay, assignment, ... 

% Under Duration, insert actual duration of test or examination and notional hours for projects and assignments.

% Under Component: Give all parts of the overall assessment, e.g. an assignment or a group of questions in a paper that must be passed to pass the course. Failing the component means a fail for the course and is reflected by the result code FCOM. The outcomes associated with a component must be identified in Section 3 above and a related body of knowledge may be linked to the outcome.  

% Under Method and Weight: Two methods of assessing student work are: 
% -	Assignment of Marks (as in traditional examinations) or 
% -	Satisfaction of Outcomes using a rubric, in which case a single mark is derived from the level of achievement against the outcomes. In such a case only a total mark should be shown. Describe the essential elements of the rubric in section 6.2, if applicable.

% Under Calculator Type, specify the calculator allowed in tests/exams. These are as follows: 0 – no calculator:
% 1- Basic scientific calculator, not programmable, limited memory, single input mathematical functions, 1 or 2 line display;
% 2 - Advanced scientific calculator: programmable, substantial memory; array functions; statistical and engineering functions built-in or loadable; multi-line/graphic display.
% 3 - Unlimited computational support.
% Types 0, 1 and 2 assume that other form of support such as mobile phones and laptop computers are not permitted.
%     Examiners should bear in mind that Type 2 calculators are potential repositories of supporting information. Type 2 calculators should be permitted or specified only in third and fourth year course examinations.

% Under Permitted Supporting Material specify what the student may have in his/her possession during the assessment.

\subsection{Assessment Methods}\label{methods}

% This was formerly Assessment Criteria. Most CBOs described the methods (I.e. how the assessment is performed), not criteria ( i.e. what the students must demonstrate).

%%%%%%%%%%%%%%%%%%%%%%%%%%%%%%%%%%%%%%%%%%%%%%%%%%%%%%%%%%%%%%%%%%%%%%%%%%%%%%%%
\section{Satisfactory Performance (SP) Requirements}\label{SP}
% This is a single generic statement applicable to all courses except perhaps Lab Project and Design II: 
For the purpose of Rule G.13 \emph{satisfactory performance in the work of the class} means attendance and completion of prescribed laboratory activities, attendance at tutorials designated as compulsory in this CB\&O, submission of assignments, writing of scheduled tests unless excused in terms of due procedure.

% A note on specific course requirements may be added, e.g. specific laboratory reports, etc.  

%%%%%%%%%%%%%%%%%%%%%%%%%%%%%%%%%%%%%%%%%%%%%%%%%%%%%%%%%%%%%%%%%%%%%%%%%%%%%%%%
\section{Teaching and Learning Process}\label{teaching}

\subsection{Teaching and Learning Approach}
% Describe the learning approach(es) expected of the student and the facilitation and support provided by the staff.

\subsection{Information to Support the Course}
% This section must clearly state where the complete knowledge base for the course that covers the content and supports fulfillment of outcomes can be found. This would include prepared notes, textbooks, papers, web references (pre-screened by lecturer).  This section may list additional reading.

\subsection{Learning Activities and Arrangements}

% < Where applicable specify the learning and teaching strategy(ies), for example from the list:
% a.	Traditional Lecture 
% b.	Demonstration-based lecture
% c.	Flipped lecture
% d.	Problem-based learning
% e.	Writing-intensive learning
% f.	Design-build-and-test	g.	Structured laboratory
% h.	Investigative laboratory
% i.	Peer group activity
% j.	Parallel small group tutorials
% k.	Small group tutorial within larger tutorial>

\subsubsection{Lectures}
% Approach to lectures

\subsubsection{Tutorials}
% Approach to tutorials; Where to find the tutorial arrangements.

\subsubsection{Project/Assignment}
% When and how project brief will be made available; when and how submission to be made; Draw attention to the School's policy on timely submission of projects as per the School’s Red Book will be enforced and must be read by the student.

\subsubsection{Laboratory}
% Specify the laboratory sessions/experiments, where to find the timetable, ...

\subsubsection{Consultation}
% Insert Lecturer’s preferred way of providing consultation time

%%%%%%%%%%%%%%%%%%%%%%%%%%%%%%%%%%%%%%%%%%%%%%%%%%%%%%%%%%%%%%%%%%%%%%%%%%%%%%%%
\section{Course Home Page}\label{web}
Further information and announcements regarding the course are posted on the course home page: \url{https://???.???.???/}

All students are expected to consult the course home page at regular intervals.

\end{document}

